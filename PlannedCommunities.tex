\documentclass[twocolumn]{article}

\usepackage{verbatim}
\usepackage{graphicx}
\usepackage{geometry}
\usepackage[firstpage]{draftwatermark}
\usepackage{cuted}
\usepackage{float}


\renewcommand{\familydefault}{\sfdefault}

\title{Planned Communities}
\date{2019-02-01}
\author{Tobias Blaser}

\geometry{
	left=2cm,
	top=2cm,
	right=2cm,
	bottom=3cm
}


\begin{document}

	\newgeometry{top=5cm,right=0cm,left=0cm,bottom=2cm}
	\pagenumbering{gobble}
	
	\SetWatermarkScale{0.75}
	\SetWatermarkText{
		\includegraphics[angle=-45]{images/cover-background.jpg}
	}
	
	\maketitle	
 	
	\begin{figure}[b!]
 		\includegraphics[
 			width=\textwidth,
 			trim={0 1cm 0 1cm},
 			clip
 		]{images/la-chaux-de-fonds/1280px-Rue_du_Progrès_in_La_Chaux-de-Fonds.jpg}
 	\end{figure}
 	
 	\restoregeometry	
	
	
	\clearpage
	\pagenumbering{arabic}
	
	\begin{strip}
	\tableofcontents
	\paragraph{Cover images}
	Map of La Chaux-de-fonds 1920\cite{MapGeoAdmin:LaChauxDeFonds1920},
	Rue du Progrès in La Chaux-de-Fonds\cite{Wikimedia:RueDuProgressLaChauxDeFonds}
	\end{strip}
	
	% TODO page break
	\clearpage
	\section{Summary}
		\subsection{Purpose}
		The purpose of this document, is to compare the growth strategies of planned cities from different countries and centuries.
		The following cities are compared:
		
		\begin{description}
			\item [La Chaux-de-Fonds] Switzerland, Europe, Planned 1794
			\item [La Plata] Argentina, South America, Planned 1882
			\item [Abuja FCT] Nigeria, Africa, Planned 1976
		\end{description}
		La Plata (Argentina), La Chaux-de-Fonds (Switzerland) and Abuja FCT (Nigeria)

		\subsection{Methodology}

		\subsection{Findings}


	\clearpage
	\section{Methodology}
	
	
	\begin{enumerate}
		\item Analyze, how the city was planned
		\item Analyze, how the city looks today
		\item Analyze, how the city grew since the plan was implemented
		\item Analyze the sustainability of the growth
			\begin{itemize}
				\item Mode shares, walkability
				\item Density
				\item Attractivity for developers
			\end{itemize}
		\item Compare the different growth strategies
	\end{enumerate}


	\clearpage
	\begin{strip}
	\section{Communities}
	
		\subsection{La Chaux-de-Fonds, Switzerland}
		\end{strip}
			% TODO La Chaux-de-Fonds world map
		
			% TODO fire 1794
			% https://www.imagesdupatrimoine.ch/notice/article/la-chaux-de-fonds-brule.html
			% images/la-chaux-de-fonds/fire-1794.jpg
		
			\subsubsection{Original plan}
			La Chaux-de-Fonds, located in the french part of Switzerland, burned down in 1794.
			This catastrophe was taken as an opportunity, to plan the future development of the city in 1841\ref{fig:map:plan-la-chaux-de-fonds-1841}.
		
			\begin{figure}[H]
				\includegraphics[width=\linewidth]{%
					images/la-chaux-de-fonds/1280px-La_Chaux-de-Fonds1841.png%
				}
				\caption{Development plan of La Chaux-de-Fonds from 1841\cite{Wikimedia:LaChauxDeFonds1841}}
				\label{fig:map:plan-la-chaux-de-fonds-1841}
			\end{figure}
			
			
			The plan designed a grid layout with rectangular blocks of around 45m x 125m\ref{fig:img:la-chaux-de-fonds-block-layout-1900}, containing one row of buildings, bordering to the street on one side and containing open space or yards on the other side.
			
			\begin{figure}[H]
				\includegraphics[width=\linewidth,trim={1cm 1cm 1cm 1cm}]{%
					images/la-chaux-de-fonds/block-layout-1900.pdf%
				}
				\caption{Block layout in La Chaux-de-Fonds 1900\cite{MapGeoAdmin:LaChauxDeFonds}. (The first map of that detail level of La Chaux-de-Fonds is available from 1897)}
				\label{fig:img:la-chaux-de-fonds-block-layout-1900}
			\end{figure}
			
			
			The city was planned for pedestrians, cyclists, horses and carts.
			It was planned as a compact city. Everything should be reachable by foot.
			
			
			\begin{figure}[H]
				\includegraphics[width=\linewidth,trim={1cm 1cm 1cm 1cm}]{%
					maps/la-chaux-de-fonds/1845.pdf%
				}
				\caption{La Chaux-de-Fonds 1945\cite{MapGeoAdmin:LaChauxDeFonds}}
				\label{fig:map:la-chaux-de-fonds-1945}
			\end{figure}
			
			Until 1870\ref{fig:map:la-chaux-de-fonds-1945}, there was not much development, the city kept their small size and only a few new blocks were developed.
			
			
			\subsubsection{Development}
			\begin{figure}[H]
				\includegraphics[width=\linewidth,trim={1cm 1cm 1cm 1cm}]{%
					maps/la-chaux-de-fonds/1900.pdf%
				}
				\caption{La Chaux-de-Fonds 1900\cite{MapGeoAdmin:LaChauxDeFonds}}
				\label{fig:map:la-chaux-de-fonds-1900}
			\end{figure}
			
			From 1870 to 1900\ref{fig:map:la-chaux-de-fonds-1900}, the city grew fast to a diameter of around 1.5kms, reaching the borders of the plan from 1841.
			1900 it was already connected by four national railway lines and streetcar lines were under construction.
			% TODO pictures from La Chaux de Fonds 1888
			% http://cdf-bibliotheques.ne.ch/bvcf/patrimoine/dossiers-thematiques/plans/Pages/reconstruction_a_1888.aspx
			
			
			\begin{figure}[H]
				\includegraphics[width=\linewidth,trim={1cm 1cm 1cm 1cm}]{%
					maps/la-chaux-de-fonds/1960.pdf%
				}
				\caption{La Chaux-de-Fonds 1960\cite{MapGeoAdmin:LaChauxDeFonds}}
				\label{fig:map:la-chaux-de-fonds-1960}
			\end{figure}
			% TODO put map from 1954 (first sprawled districts)
			
			From 1900 to 1950\ref{fig:map:la-chaux-de-fonds-1960}, the city continued growing fast, continuing the pattern the plan from 1841 had defined. The city reached a diameter of around 2kms. 
			Also the development reached the station and continued on the other side, but kept growing only at the base of the valley.
			The city still was compact, pedestrian, cyclist and streetcar oriented.
			
			
			\begin{figure}[H]
				\includegraphics[width=\linewidth,trim={1cm 1cm 1cm 1cm}]{%
					images/la-chaux-de-fonds/sprawling-development-1960.pdf%
				}
				\caption{Sprawling development in La Chaux-de-Fonds around 1960\cite{MapGeoAdmin:LaChauxDeFonds}}
				\label{fig:map:la-chaux-de-fonds-sprawling-development-1960}
			\end{figure}
			
			From 1950\ref{fig:map:la-chaux-de-fonds-sprawling-development-1960} the development started crawling up the hills around La Chaux-de-Fonds.
			New districts with big yards and lower density were constructed fast.
			The development did not follow anymore the planned pattern from 1841.
			
			
			Because of the growing popularity of the automobile, the street network was intensively extended and streets were paved.
			Also industrial parks in the outlying areas were constructed.
			
			Since 1980 Switzerland counts with land-use planning laws and zone planning. Zone planning restricts development outside of development zones. Anyhow increased the development area in Switzerland since 1950 by 100\%.
			The restrictive zone planning slowed the sprawling development in La Chaux-de-Fonds down.
			
			
			\subsubsection{Today}
			
			Today\ref{fig:map:la-chaux-de-fonds-2018}, La Chaux-de-Fonds is around 5kms long and 2.5km wide. Several districts at the hillside provide expensive apartments and houses to solvent inhabitants.
			% TODO Picture of hillside constructions
			
			\begin{figure}[H]
				\includegraphics[width=\linewidth,trim={1cm 1cm 1cm 1cm}]{%
					maps/la-chaux-de-fonds/2018.pdf%
				}
				\includegraphics[width=\linewidth,trim={1cm 1cm 1cm 1cm}]{%
					images/la-chaux-de-fonds/areal-2019.pdf%
				}
				\caption{La Chaux-de-Fonds 2018\cite{MapGeoAdmin:LaChauxDeFonds}}
				\label{fig:map:la-chaux-de-fonds-2018}
			\end{figure}
			
			Because of it's general high density, La Chaux-de-Fonds counts with good transit\ref{fig:la-chaux-de-fonds-public-transport} options. Even if the streetcar lines do not exist anymore, an extensive bus network reaches also districts up in the hills.
			
			
			% TODO Haltestellen karte map geo admin
			
			\begin{figure}[H]
				\includegraphics[width=\linewidth]{%
					images/la-chaux-de-fonds/public-transport-plan.jpg%
				}
				\caption{La Chaux-de-Fonds public transport network\cite{TransN:LaChauxDeFonds}}
				\label{fig:la-chaux-de-fonds-public-transport}
			\end{figure}
			
			
			\subsubsection{Wrap-up}
			% TODO growth ring map
			La Chaux-de-Fonds was planned as a pedestrian city in it's time.
			The arrival of the automobile caused the construction of low density districts.
			But the geographical location in the valley and the zone restrictions since 1980 limited the city from sprawling out.
			La Chaux-de-Fonds ist still a pedestrian friendly and compact city and counts with great public transport services.
			
			
		\clearpage
		\begin{strip}
		\subsection{La Plata, Argentina}
		\end{strip}
		
			% TODO La Plata world map
			
			\subsubsection{Original Plan}
			Buenos Aires, the biggest and most powerful city of Argentina, was national and state capital. To reduce the power and separate political interests, a new capital for the state Buenos Aires was planned and funded in 1882.
			The city was located 10kms inland of the river coast at the railway line from Ensenada to Buenos Aires, around 50kms away from the capital.
			
			\begin{figure}[H]
				\includegraphics[width=\linewidth]{%
					images/la-plata/planolaplata.jpg%
				}
				\caption{La Plata grid plan, designed by Pedro Benoit, 1882\cite{RecoletaCemetery:PedroBenoit}}
				\label{fig:img:plan-la-plata-1882}
			\end{figure}
			
			The plan shows a 38 x 38 block grid, criss-crossed by diagonals and most of the city’s government buildings and major churches.
			
			The diagonal boulevards are have been preserved and are a characteristic of La Plata until today.
			
			Many suburbs were founded together with the city:
			- Los Hornos
			- Ringuelet
			- San Carlos
			
			
			Some existed before and were merged, when the city was founded:
			- Tolosa
			- Altos de San Lorenzo
			
			% https://es.wikipedia.org/wiki/Gran_La_Plata
			
			
			\subsubsection{Development}
			
			
			\begin{figure}[H]
				\includegraphics[width=\linewidth]{%
					maps/la-plata/1901-buenosaires-laplata.jpg%
				}
				\caption{Buenos Aires and La Plata 1901\cite{RiviereDeLaPlata}}
				\label{fig:map:buenosaires-la-plata-1901}
			\end{figure}
			% https://http2.mlstatic.com/plano-1901-buenos-aires-la-plata-D_NQ_NP_615971-MLA28714158588_112018-F.webp
			
			
			% TODO Images from La Plata from 1900
			% La Plata 1885 http://construirtv.com/wp-content/uploads/2015/10/ciudad-mirando-al-oeste_Bradley.jpg
			
			In the first 20 years, the city grew fast, government buildings were constructed, streetcar lines were opened, the university was founded.
			
			1900: Around 70'000 inhabitants.
			
			1915, the city already had more than 130'000 inhabitants.
			% 1932 Palacio Municipal
			% https://es.wikipedia.org/wiki/Archivo:Palacio_Municipal_y_Eje_Fundacional_de_La_Plata_(1932).jpg
			
			% Historic fotos of la plata
			% https://www.taringa.net/+imagenes/historia-y-fotos-de-la-catedral-de-la-plata_z48jx
			
			% http://www.gotakey.com/blog/14/04/2016/porque-la-plata-tiene-tantas-diagonales/
			
			
			
			\begin{figure}[H]
				\includegraphics[width=\linewidth,trim={1.5cm 1.5cm 1.5cm 1.5cm},clip]{%
					images/la-plata/la_plata_1952.jpg%
				}
				\caption{La Plata 1952\cite{MOSP:InvestigacionHistorica}}
				\label{fig:map:la-plata-1952}
			\end{figure}
			
			\begin{figure}[H]
				\includegraphics[width=\linewidth]{%
					images/la-plata/plaza-central-1940.jpg%
				}
				\caption{Plaza Mariano Moreno of La Plata 1940\cite{Blogspot:Arqruotolo:la-plata-o-la-geometria-hecha-espacio}}
				\label{fig:img:la-plata-1940}
			\end{figure}
			1952: In Tolosa, Los Hornos and Villa Elvira, the city already grew outside of the layout from 1882.
			
			1950: Around 300'000 inhabitants.
			The city is bursting out of its design pattern. 
			
			% http://bibliotecadigital.uns.edu.ar/scielo.php?script=sci_arttext&pid=S1852-42652013001100002&lng=es
			% http://bibliotecadigital.uns.edu.ar/pdf/reuge/v22n1/v22n1a02.pdf
			
			
			
			2000: More than 650'000 inhabitants.
		
			
			
			\subsubsection{Today}
			
			Today La Plata has around 700'000 inhabitants. The city spreads over an aeria\ref{fig:map:la-plata-2019} which has multiple times the size of the original plan from 1882.			
			
			\begin{figure}[H]
				\includegraphics[width=\linewidth]{%
					maps/la-plata/2019.jpg%
				}
				\includegraphics[width=\linewidth]{%
					images/la-plata/2019-aerial.jpg%
				}
				\caption{La Plata 2019\cite{OpenStreetMap:LaPlata}}
				\label{fig:map:la-plata-2019}
			\end{figure}
			
			The last streetcar line was closed down in 1966, today La Plata counts with 23 bus lines\ref{fig:map:la-plata-transit} connecting the city center and the suburbs.
			The outer suburbs are not well connected by public transit and mostly car-dependent.
			
			\begin{figure}[H]
				\includegraphics[width=\linewidth]{%
					maps/la-plata/transit-lines.jpg%
				}
				\caption{La Plata public transportation\cite{OpenStreetMap:LaPlata}}
				\label{fig:map:la-plata-transit}
			\end{figure}
			
			% TODO cycle map
			
			
			\subsubsection{Wrap-up}
			
			
			
		\clearpage
		\begin{strip}
		\subsection{Abuja, Nigeria}
		\end{strip}
		
			% TODO Abuja world map
			
			% http://www.newtowninstitute.org/spip.php?article1050
			
			\subsubsection{Original Plan}
			
			Abuja was planned in 1970 as new capital for Nigeria, as replacement of Lagos.
			The center of the city is the Central Business District, an area with high buildings and commercial areas.
			The original plan is quite auto oriented. Even in the CBD, the blocks are large and pedestrian infrastructure is missing. The roads and highways were planned for high capacity of vehicles.
			First development phase contained the districts inside of the first highway ring.
			
			
			\begin{figure}[H]
				\includegraphics[width=\linewidth]{%
					maps/abuja/abuja-development-map.jpg%
				}
				\caption{Abuja development plan, 1970\cite{NairalandForum:AbujaMap}}
				\label{fig:map:abuja-development-plan}
			\end{figure}
			
			\begin{figure}[H]
				\includegraphics[width=\linewidth]{%
					maps/abuja/abuja-councils-map.jpg%
				}
				\caption{Six area councils of Abuja\cite{ResearchGate:SixCouncils}}
				\label{fig:map:abuja-six-area-councils}
			\end{figure}
			
			
			
			\subsubsection{Development}
			
			- Fast development of slums
			- Government lost development control
			- Much low density development in planned districts, high density development in the slums
			
			\subsubsection{Today}
			
			Abuja is still growing fast. Meanwhile development inside the planned area is following the master plan, slums are growing outside the planned areas. The government is not able anymore, to control that development.
			
			\begin{figure}[H]
				\includegraphics[width=\linewidth]{%
					images/abuja/2019-aerial.png%
				}
				\caption{Abuja satellite foto, 2019\cite{Satellites.pro:Abuja}}
				\label{fig:images:abuja-aerial-2019}
			\end{figure}
			
			Slums are also filling open areas inside the plannes area and are growing fast.
			
			\begin{figure}[H]
				\includegraphics[width=\linewidth]{%
					images/abuja/slums.png%
				}
				\caption{Slums in Abuja (bottom), along with planned development (top), 2019\cite{Satellites.pro:Abuja}}
				\label{fig:images:abuja-slums}
			\end{figure}
			
			Most of the districts in the center count with low density.
			
			\begin{figure}[H]
				\includegraphics[width=\linewidth]{%
					images/abuja/Map-showing-Abujas-built-up-density-PBA-for-each-MP-sector-The-color-palette-ranges.png%
				}
				\caption{Abuja's built-up density (PBA), 2015\cite{ResearchGate:AbujaDensity}}
				\label{fig:images:abuja-density}
			\end{figure}
			
			In the CBD, several bus lines are connecting places. Some years ago, the first railway line, connecting the airport with the city center, was finished.
			The rest of the rail network is yet outstanding. The national trains arrive at the main station outside of Abuja, but do not get yet to the city center.
			
			\begin{figure}[H]
				\includegraphics[width=\linewidth]{%
					maps/abuja/public-transport.jpg%
				}
				\caption{Abuja future rail transport network\cite{NairalandForum:AbujaMap}}
				\label{fig:map:abuja-future-rail-network}
			\end{figure}			
			
			
			\subsubsection{Wrap-up}
			
			


	\clearpage
	\section{Conclusion}

	
	\begin{comment}
	https://www.latex-tutorial.com/tutorials/table-of-contents/
	https://en.wikibooks.org/wiki/LaTeX/
	
https://en.m.wikipedia.org/wiki/Planned_community

https://de.wikipedia.org/wiki/La_Chaux-de-Fonds
Batavia, Indonesia, 17th century, https://en.m.wikipedia.org/wiki/Batavia,_Dutch_East_Indies
Ashdod, Israel, 1956,https://en.m.wikipedia.org/wiki/Ashdod
Kyoto, Japan, 794, https://en.m.wikipedia.org/wiki/Kyoto
Navi Mumbai, India, 1960, https://en.m.wikipedia.org/wiki/Navi_Mumbai
Adelaide, Australia, 1836, https://en.m.wikipedia.org/wiki/Adelaide
Washington D.C., USA, 1790, https://en.m.wikipedia.org/wiki/Washington,_D.C.

two figures in one line
\begin{figure}[h!]
  \centering
  \begin{subfigure}[b]{0.4\linewidth}
    \includegraphics[width=\linewidth]{coffee.jpg}
    \caption{Coffee.}
  \end{subfigure}
  \begin{subfigure}[b]{0.4\linewidth}
    \includegraphics[width=\linewidth]{coffee.jpg}
    \caption{More coffee.}
  \end{subfigure}
  \caption{The same cup of coffee. Two times.}
  \label{fig:coffee}
\end{figure}

\begin{itemize}
\item \blindtext
\item \blindtext
\end{itemize}
\begin{enumerate}
\item \blindtext
\item \blindtext
\end{enumerate}
\begin{description}
\item [Ant] \blindtext
\item [Elephant] \blindtext
\end{description}

@BOOK{DUMMY:1,
AUTHOR="John Doe",
TITLE="The Book without Title",
PUBLISHER="Dummy Publisher",
YEAR="2100",
}


	\end{comment}
	
	\clearpage
	\begin{strip}
		\begin{appendix}
			\bibliography{references} 
			\bibliographystyle{ieeetr}
		
			\listoffigures
		
			\listoftables
		\end{appendix}
	\end{strip}
\end{document}


